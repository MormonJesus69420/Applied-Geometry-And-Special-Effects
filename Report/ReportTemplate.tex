% Article type supporting font formatting
\documentclass[a4paper,12pt]{extarticle}

% Define .tex file encoding
\usepackage[utf8]{inputenc}

% Norwegian language support
%\usepackage[norsk]{babel}     

% Indent first paragraph in section
\usepackage{indentfirst}

% Allows mathbb in tex file
\usepackage{amsfonts} 

% Margin defining package
\usepackage{geometry}         
\geometry{a4paper
  ,margin=2cm
}

% Allows easy writing of algorithms
\usepackage[ruled]{algorithm2e}

% Math related tools and functions
\usepackage{mathtools} 

% For quotations
\usepackage{csquotes}

% Better bibliography 
\usepackage[round]{natbib}

% For use of graphics in document
\usepackage{graphicx}         

% Allows multi-line comments in tex file
\usepackage{verbatim} 

% Allows more control over tables
\usepackage{tabulary}

% Ads ul which allows line breaks while underlining text.
\usepackage{soul}      

% Allows math in tex file 
\usepackage{amsmath}          

% Allows math symbols in tex file
\usepackage{amssymb}          

% Allows use of physics shortcut functions
%\usepackage{physics}          

% Verbatim env with LaTeX commands
\usepackage{alltt}            

% Allows \begin{figure}[H]
\usepackage{float}

% Adds labeling list to the report
\usepackage{scrextend}
\addtokomafont{labelinglabel}{\ttfamily}

% Necessary for defining colours
\usepackage{xcolor}            
\definecolor{linkgreen}{rgb}{0,.5,0}
\definecolor{linkblue}{rgb}{0,0,.5}
\definecolor{linkred}{rgb}{.5,0,0}
\definecolor{blue}{rgb}{.13,.13,1}
\definecolor{green}{rgb}{0,.5,0}
\definecolor{red}{rgb}{.9,0,0}

% Hyperlinks in document
\usepackage{hyperref}  
\hypersetup{
  colorlinks=true,     % True for colored links
  linktoc=all,         % True for table of contents links
  linkcolor=linkblue,  % Colour for links
  urlcolor=linkgreen,  % Colour for URLs
  citecolor=linkred    % Colour for citations
}

% Listing package for code examples
\usepackage{listings}         
\lstset{
  language=C++,                % Set language to C++
  showspaces=false,            % Don't show space chars
  showtabs=false,              % Don't show tab chars
  breaklines=true,             % Break long lines of code
  showstringspaces=false,      % Don't show spaces in strings
  breakatwhitespace=true,      % Break at white space only
  commentstyle=\color{green},  % Set colour for comments
  keywordstyle=\color{blue},   % Set colours for keywords
  stringstyle=\color{red},     % Set colour for strings
  basicstyle=\ttfamily,        % Set basic style
  tabsize=2                    % Set tabsize
}

% Makes matrices look square-ish
\renewcommand*{\arraystretch}{1.5}

% Allows number referencing the last page, used in footer
\usepackage{lastpage}

% Allows editing of header and footer data
\usepackage{fancyhdr}
\fancypagestyle{plain}{
  \fancyhf{}
  \renewcommand{\headrulewidth}{0pt}
  \rfoot[R]{\footnotesize Page \thepage\ of \pageref{LastPage}}
}
\pagestyle{fancy}
\fancyhf{}
\chead{\footnotesize D.A. Salwerowicz, From B-Spline to GERBS Surface, curve manipulation and special effects}
\rfoot{\footnotesize Page \thepage\ of \pageref{LastPage}}
\setlength{\headheight}{20pt}
\setlength{\footskip}{20pt}

% Allows editing of section headers
\usepackage{titlesec}
\titleformat*{\section}{\normalsize\bfseries}
\titleformat*{\subsection}{\normalsize\itshape\bfseries}
\titleformat*{\subsubsection}{\normalsize\bfseries}

% Change look and feel of abstract
\usepackage{abstract}
\setlength{\absleftindent}{0mm}
\setlength{\absrightindent}{0mm}
\renewcommand*\abstractname{\flushleft\normalsize\textbf{Abstract}\hfill}

% Change caption size and style
\usepackage[font=footnotesize,labelsep=period]{caption}

% Change date style
\usepackage[yyyymmdd]{datetime}
\renewcommand{\dateseparator}{-}

% Referencing, last for compatibility reasons
\usepackage[noabbrev]{cleveref}

%%%%%%%%%%%%%%%%%%%%%%%%%%%%%%%%%%%%%%%
%%      Title, Author, and Date      %%
%%%%%%%%%%%%%%%%%%%%%%%%%%%%%%%%%%%%%%%
\title{From Simple B-Spline to GERBS Surface, or How I Stopped Sleeping and Learned to Hate My Life}
\author{D. A. Salwerowicz}
\date{\parbox{\linewidth}{\centering
    \textit{\small UiT - The Arctic University of Norway, P.O. Box 385, N-8505 Narvik, Norway}\endgraf\bigskip
    \small Submitted \today
}}
\providecommand{\keywords}[1]{\flushleft\textit{\small{Keywords:}} #1}

%%%%%%%%%%%%%%%%%%%%%%%%%%%%%%%%%%%%%%%
%%           Start document          %%
%%%%%%%%%%%%%%%%%%%%%%%%%%%%%%%%%%%%%%%
\begin{document}
  
%%%%%%%%%%%%%%%%%%%%%%%%%%%%%%%%%%%%%%%
%%   Create the main title section   %%
%%%%%%%%%%%%%%%%%%%%%%%%%%%%%%%%%%%%%%%
\maketitle

%%%%%%%%%%%%%%%%%%%%%%%%%%%%%%%%%%%%%%%
%%      Abstract for the report      %%
%%%%%%%%%%%%%%%%%%%%%%%%%%%%%%%%%%%%%%%
\noindent\rule{\linewidth}{.5pt}
\begin{abstract} 
\verb|//TODO|

\keywords{Parametric Curves; B-Splines; Curve Blending; GERBS}
\end{abstract}
\rule{\linewidth}{.5pt}

%%%%%%%%%%%%%%%%%%%%%%%%%%%%%%%%%%%%%%
%%  The main content of the report  %%
%%%%%%%%%%%%%%%%%%%%%%%%%%%%%%%%%%%%%%

\section{Introduction}
This project revolves around implementing various geometric objects using GMlib as a basis. Each of these object builds on previous therefore I will describe them seperately. Main focus of this project was to implement a GERBS curve and use affine transformations to create a dynamic animation.

\subsection{B-Spline curve}
Most basic object that I have implemented is a third degree, fourth order B-Spline curve using both a vector of \emph{control points} and \emph{sampling}.

\subsection{Blending curve}
A more advances object implemented by me is a blending curve that takes in two curves and blends them into one curve using any percent of the original curves it wants. So I can use 50\% or 75\% of these curves and blend the rest.

\subsection{GERBS curves and surfaces}
Last and most complex objects implemented are a GERBS curve and surface, where GERBS stands for \emph{Generalized Expo-Rational B-Spline}. These objects are build out of local curves and surfaces that are blended together to form a curve/surface, instead of simple control points.

\section{Material \& Methods}
Here I will describe the most important tools and methods used during development of my project. All of the theory in this section is based on book "Blending technics[sic] for Curve and Surface Constructions" written by Arne Laks\aa \,\citep{Laksa2012}.

\subsection{GMlib}
It's important to mention that I have based my project and work on the Geometric Modeling library also known as GMlib. It was developed at Arctic University of Norway, UiT Narvik. It handles a lot of necessary background calculation and rendering of objects on screen for me. This way I only needed to focus on implementing necessary functionality for calculating and showing B-Spline curves and other objects.

\subsection{B-Spline Curves}
\subsubsection{B-Spline from control points}
One can create a B-Spline curve from a vector of \emph{control points} $\vec{c}$, which is then used to create a \emph{knot vector} $\vec{t}$. For any given B-Spline a knot vector will be of length $n+k$ where $n$ is length of control point vector and $k$ is the order. So in my case knot vector will have lenght 12 if I give it 8 control points. Resulting knot vector will look like one shown in \cref{eq:KnotVectorBSpline}.

\begin{equation}
\vec{t}= [0,0,0,0,1,2,3,4,5,5,5,5]
\label{eq:KnotVectorBSpline}
\end{equation}

This B-Spline will be a \emph{Clamped B-Spline} as its first $k$ and last $k$ knots are equal to each other. Then I can use formula \cref{eq:BSplineFormula} to calculate my BSpline.

\begin{equation}
c(t)= \sum_{i=1}^{n} c_i b_{k,i}(t), \quad t \in \left[t_d,t_n\right]
\label{eq:BSplineFormula}
\end{equation}

Here ${c_i}_{i=1}^n$ is a vector of controll point and $b_{k,i}(t)$ are basis functions that are defined by $t$ \citep[Chap 5.5.1]{Laksa2012}. \citep[Chap 5.5.3]{Laksa2012} describes in depth how we calculate these basis functions. Resulting basis functions are shown in \cref{eq:BasisFunctions}, there are four of them, since the order of my B-Spline is $4$.

\begin{equation}
\begin{split}
b_{k-3,i} &= \left[ \left( 1-w_{1,i}(t) \right) \left( 1-w_{2,i-1}(t) \right) \right] \left[ 1-w_{3,i-2}(t) \right]\\
b_{k-2,i} &= \left[ \left( 1-w_{1,i}(t) \right) \left( 1-w_{2,i-1}(t) \right) \right] \left[ 1-w_{3,i-1}(t) \right]\\ &+ \left( \left[ \left( 1-w_{1,i}(t) \right) \left( w_{2,i-1}(t) \right) \right] + \left[ w_{1,i}(t) \left( 1-w_{2,i}(t) \right) \right] \right) \left[ 1-w_{3,i-t}(t) \right]\\
b_{k-1,i} &= \left( \left[ \left( 1-w_{1,i}(t) \right) \left( w_{2,i-1}(t) \right) \right] + \left[ w_{1,i}(t) \left( 1-w_{2,i}(t) \right) \right] \right) \left[ w_{3,i-1}(t) \right]\\
b_{k,i} &= w_{1,i}(t) w_{2,i}(t) w_{3,i}(t)
\label{eq:BasisFunctions}
\end{split}
\end{equation}

$w_{d,i}(t)$ is a so called linear translation and scaling function used in B-Splines to translate the $t$ to a range $[0,1]$, and is defined in \cref{eq:WFunction}.

\begin{equation}
  w_{d,i}(t) = 		
  \begin{cases}
    \frac{t-t_i}{t_{i+d} - t_i}, & \text{if } t_i \leq t \leq t_{i+d},\\
    0, & \text{otherwise.}
  \end{cases}
  \label{eq:WFunction}
\end{equation}

Thus using these basis functions I am able to evaluate my B-Spline and draw it on screen which is shown in the first simulation in my program.

\subsubsection{Sampled B-Spline}
Another way of creating a B-Spline is by sampling points along an existing curve and then using least square method to create control points from a set of points, $p$. It is quite usefull as it lets us approximate any kind of freeform curve to a high degree of accuracy.

Knot vector is created the same way as before, I just use the provided $n$ to create it. However in order to calculate a vector of control points I have to perform matrix calculations. I first start by calculating an $A$ matrix. Values in it are defined by formula in \cref{eq:LeastSquare}.

\begin{equation}
\partial t = \frac{t_n - t_d}{m-1}
\label{eq:LeastSquare}
\end{equation}

Where $m$ is the dimension of $p$. Algorithm used to calculate values in $A$ matrix is as shown in \cref{alg:CalculateA}.

\begin{algorithm}
  \KwData{$m, n, d$}
  \KwResult{$A$ matrix of basis functions used in calculating control points.}
  Create an empty 2D matrix with $\text{dim}_1 = n$ and $\text{dim}_2 = m$\;
  Calculate $\partial t$ using \eqref{eq:LeastSquare}\;
  \For{$a_1 = 0;\ a_1 < m;\ ++a_1$}{			
    Find $i$ using $t = t_d + a_1 \cdot \partial t$\;
    Find $\vec{b}$ using \eqref{eq:BasisFunctions} with $k=t_d + a_1 \partial t$\;
    \For{$a_2 = i-d;\ a_2 \leq i;\ ++a_2$}
    {
      $A_{a_1,a_2} = b_{a_2-i+d}$\;
    }
  }
  \caption{Calculating values for matrix A}
  \label{alg:CalculateA}
\end{algorithm}

Using this algorithm we end up with a matrix that has non-zero values along its diagonal, and is zero elsewhere. Using this matrix and property: $$Ac = p$$ I can easily solve for $c$. However since $A$ is an asymetrical matrix I need to multiply it with itself transposed. So I multiply both side of the equation with $A^T$ and then substitute $A^TA=D$, lastly I multiply both sides of the equation with inverse of $D$ and get:

\begin{equation}
c=D^{-1}b
\end{equation}

After calculating this, I get a control point vector that I can then use to draw my B-Spline.

\subsubsection{Curve blending}
Blended curve is simply a curve that is result of applying a \emph{B-Function} to two or more curves. B-function per definition is a \emph{permutation function}, where $B(0)=0$ and $B(1)=1$, it is as well \emph{monotone}, and can be symetric if it satisfies $B(t)+B(1-t)=1$. \citep[Chap 6.1]{Laksa2012}

There are several viable blending functions and ways of blending two curves. In my implementation I have chosen blending function shown in \cref{eq:BFunction} which is a polynomial function of first order. It is simple to calculate and derivate. My program gives possibility to choose how much of the original curve is used before the programs starts to blend the two together. \citep[Chap 6.2.2]{Laksa2012}

\begin{equation}
B(t)= 3t^2 - 2t^3
\label{eq:BFunction}
\end{equation}

To evaluate a curve at blending point I use $c_3(t)$, defined by formula described in \cref{eq:BlendedCurve}. It uses $x$ as a blending point so that one can choose how much of original curves is used before blending occurs.

\begin{equation}
c_3(t)= c_1(t) + B\left( \frac{t-x}{1-x} \right) (c_2(t-x) - c_1(t)), \quad x \leq t < 1 \text{ and } x \in \left\langle 0,1 \right\rangle.
\label{eq:BlendedCurve}
\end{equation}

Total curve is then defined by \cref{eq:TotalBlendedCurve}.

\begin{equation}
f(t) =
\begin{cases}
c_1(t), & \text{if }0 \leq t < x,\\
c_3(t), & \text{if } x \leq t < 1,\\
c_2(t-x) & \text{if }1 \leq t \leq 1 + x.
\end{cases}
\label{eq:TotalBlendedCurve}
\end{equation}

Which gives me a $C_1$ smooth curve in domain $[0,1+x]$.

\subsubsection{GERBS Curve}
\emph{General Expo-Rational B-Spline curve} or GERBS curve for short is a more advanced type of a B-Spline curve that instead of simple control points uses \emph{control curves} or \emph{local curves}. This gives one much more control over the resulting curves as one can not only change the position of control curve, but also its orientation by flipping, rotating or shrinking it. This control comes at a cost, as it is much more expensive and data demanding to evaluate these curves and they are much less numerically stable than simple B-Splines. Final curve is a result of blending these local curves into one, so one can easily see that they build up on two previous curve types. This blending and need for considering orientation of local curves is what makes it harder to work with GERBS.

One also needs to take into consideration whether a curve is open or closed as the local curves span several knots and each point on the curve is defined by two curves. In case a curve is closed then I need to adjust the first and last knot vector by formulas described in \cref{eq:Knotting}, and the last subcurve is the same as the first one.

\begin{align}
\begin{split}
t_0&=t_d-(t_{n+d}-t_n)\\
t_{n+k}&=t_{n+d}+(t_k-t_d)
\label{eq:Knotting}
\end{split}
\end{align}

Besides that knot vector is made quite similar to the way that it is made for B-Splines, however it must start and end at values defined by model curve. In my case the model curve I chose starts at $0$ and ends at $2\pi$ and is known as one of the many heart curves, but it gives the nicest looking heart. It is defined by \cref{eq:HeartCurve} and I found this definition in \citep{Weisstein2018}.

\begin{align}
\begin{split}
x &= 16 \sin^3(t) \\
y &= 13 \cos(t) - 5 \cos(2t) - 2 \cos(3t) - \cos(4t), \quad t \in [0,2\pi]
\label{eq:HeartCurve}
\end{split}
\end{align}

After creating a knot vector it's relatively easy to create local curves based on $t$. In my case there are 10 subcurves and as such the knot vector contains 13 knots. In order to evaluate the curve I simply blend the two curves that define each point.

After my curve is created I apply simple affine transformations to each local curve creating an animation that looks like a beating heart. Transformations used here are translation and rotation.

\subsubsection{GERBS Surface}
GERBS surfaces are similar to GERBS curves as they are also effect of blending of local surfaces, however now there are 4 surfaces defining each point on the surface and it requires two knot vectors to define it. It also requires that I check if the surface is closed in opposite direction of the knot vector so that I can adjust the knot vector the same way I did for the GERBS curve. 

Instead of a vector of subcurves, a GERBS surface uses a matrix of subsurfaces. creating it is not easy as subsurfaces in last column and row must be created or copied from first column/row depending on whether or not the model surface is closed or not.

In order to show that my code is working properly I have created three GERBS surfaces, one open, one closed in one direction, and one closed in both directions. They are as follows a plane, a cylinder and a torus.

Evaluating GERBS surface is understandably hard, general formula for GERBS surfaces is defined in \cref{eq:GERBSFormula}

\begin{equation}
s(u,v)= \sum_{i=0}^{n_1-1} \left( \sum_{j=0}^{n_2-1} c_{i,j}(u,v) b_j(u) \right) b_i(v), \quad b_j(u) = B \circ w_{1,j}(u), b_i(v) = B \circ w_{1,i}(v).
\label{eq:GERBSFormula}
\end{equation}

Which in turn can be turned into:

\begin{equation}
S(u,v)=
\begin{pmatrix}
1-b_j & b
\end{pmatrix}
\begin{pmatrix}
c_{i-1,j-1} & c_{i,j-1}\\
c_{i-1,j} & c_{i,j}
\end{pmatrix}
\begin{pmatrix}
1-b_i \\
b
\end{pmatrix}
\end{equation}

Thus getting general formulation for position of each point on the surface:
\begin{equation}
  S(u,v)= (1-b_i) (1-b_j) c_{i-1,j-1} + (1-b_i) b_j c_{i-1,j} + b_i (1-b_j) c_{i,j-1} + b_i b_j c_{i,j}
  \label{eq:GERBSGeneral}
\end{equation}

I also need to calculate derivatives of \cref{eq:GERBSGeneral}, $S_u$ $S_v$ and $S_{uv}$, and put them all in a matrix to define all the information needed for every point on the surface to get proper shading and lighting on the surface.

\begin{equation}
\begin{pmatrix}
pos & S_v\\
S_u & S_{uv}
\end{pmatrix}
\end{equation}

This matrix is calculated and recorded for each point on the surface.
 
\section{References}
\begingroup
\def\section*#1{}
\bibliographystyle{apalike}
\bibliography{References}
\endgroup
\end{document} 